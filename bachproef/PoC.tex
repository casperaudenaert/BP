\chapter{\IfLanguageName{dutch}{Proof of Concept}{Proof of Concept}}%
\label{ch:PoC}

Dit deel van het onderzoek presenteert een proof of concept gericht op welk framework er gebruik zal worden om machine learning pipelines lokaal uit te voeren, dat gebruikt kan worden in de cursus ``Machine Learning Operations''.
Het doel van deze proof of concept is de efficiëntie en bruikbaarheid van deze frameworks te onderzoeken aan de hand van een lokale opstelling met de gekozen frameworks. Hierbij zal de uitvoering van het framework alsook de resultaten en bevindingen hiervan besproken worden.
Hierna wordt, er aan de hand van de resulaten een conclusie genomen over het gekozen framework.

\section{Ontwikkeling van de Machine Learning pipeline}
In dit onderdeel wordt de opbouw van de Machine Learning pipeline besproken, dit zal er voor zorgen dat het duidelijk is hoe de pipeline in elkaar zit.
De Machine Learning pipeline dat gebruikt gaat worden is die van Labo 3 van de cursus ``Machine Learning Operations''. Deze pipeline is ontworpen om aan de hand van Image Classification het verschil te kunnen herkennen tussen fotos van ``Appelsienen'' en ``Appels'', zoals besproken in de stand van zaken heeft deze pipeline verschillende onderdelen:

\begin{itemize}
    \item Preprocessing
    \item Training
    \item Evaluatie
\end{itemize}

De volgende hoofdstukken gaan dieper in op elk onderdeel van de pipeline en zal de werking ervan uitleggen, zoals het diagram 1.4.2 aantoont word elk onderdeel achter elkaar uitgevoerd.

\subsection{Preprocessing}

\subsection{Training}
\subsection{Evaluatie}

\section*{MLflow}
Elk van deze Proof of Concepts bevat MLflow zoals besproken in heeft dit framework heel wat functionaliteiten, voor deze Proof of Concepts gaan we vooral de parameters van het model, de resultaten van de training en de resultaten van de evaluatie bijhouden.
Dit noemt Experiment Tracking, en stelt ons in staat om experimenten bij te houden.

Om van MLflow gebruikt te kunnen maken moeten we het eerst importeren hieronder is een codevoorbeeld hoe dit gebeurd:

Nadata MLflow geimporteerd is moeten we alle variablen veranderen zodat deze de resulaten naar de juiste server worden gestuurd en de juiste resulaten worden bijgehouden.
De varaiablen die we moeten bijhouden zijn als volgt:

\begin{itemize}
    \item **Parameter Logging:** Het commando hiervoor is ``mflow().log_param''. Dit houd alle parameters bij van het model, zoals het aantal epochs of de batchgrootte.
\end{itemize}




