%%=============================================================================
%% Methodologie
%%=============================================================================

\chapter{\IfLanguageName{dutch}{Methodologie}{Methodology}}%
\label{ch:methodologie}

% TODO: geen sectie zonder inleiding ervoor schrijven + geen subsectie zonder section
% TODO: hang deze verschillende fasen aan elkaar met een verhaal, nu stan ze redelijk los van elkaar: dit heb je ook nog niet gedaan
% TODO: een methodologie schrijf je in de voltooid tegenwoordige tijd, tenzij iets nog bezig is (dan onvoltooid tegenwoordige tijd)

Voor de start van dit onderzoek is het noodzakelijk om een basiskennis te verwerven over de verschillende concepten en de verschillende Machine Learning frameworks die aan bod zullen komen. De volgende secties beschrijven de fasen die werden uitgevoerd voor dit onderzoek, verbonden door een logisch verhaal.
\section{Literatuurstudie}
Een eerste fase in dit onderzoek was het uitvoeren van een literatuurstudie. Deze literatuurstudie leidde tot een oplijsting van verschillende frameworks en libraries die gebruikt zullen worden in dit onderzoek, evenals de uitleg van verschillende termen en de werking van bepaalde technologieën. De frameworks en libraries worden gekozen op basis van een literatuurstudie en een requirementsanalyse. Deze worden dan nauwkeurig beschreven in de stand van zaken. De beschrijving bevat wat elk framework is, met daaropvolgend de voor- en nadelen ervan. Dit is gebaseerd op de officiële documentatie en verschillende onderzoekstudies.
De literatuurstudie zorgt voor een longlist waaruit enkele frameworks en libraries worden gekozen om een Proof of Concept op te stellen. Op basis van voorgedefinieerde functionele en niet-functionele vereisten zal vervolgens een geschikt framework of library worden gekozen.
\section{Voorbereiding Proof of Concept}
Vooraleer er gestart kan worden aan de Proof of Concept moet er een goed begrip worden verkregen van de gekozen frameworks en libraries.
\subsection{Kennis van frameworks en libraries}
Een goed begrip van de gekozen frameworks en libraries is belangrijk voor een goede uitwerking van de Proof of Concept. Hieronder volgt een lijst met de belangrijkste aspecten:
\begin{itemize}
  \item Documentatie: De documentatie van het gekozen framework of library biedt uitleg over de werking en de functionaliteiten ervan.
  \item Syntax: Elk framework of library heeft zijn eigen syntax; het is belangrijk om deze syntax goed te begrijpen en te kunnen gebruiken.
  \item Functionaliteiten: Voor elk framework worden de functionaliteiten opgelijst.
\end{itemize}
\section{Proof of Concept}
Vervolgens werden verschillende Proof of Concepts opgesteld om te onderzoeken of de onderzoeksvraag en deelvragen uit de inleiding beantwoord kunnen worden. De Proof of Concepts zullen bestaan uit de code van Labo 3 uit het opleidingsonderdeel ``Machine Learning Operations'' die wordt omgezet naar de gekozen frameworks.
\section{Machine Learning pipeline}
Er werd gekeken naar de originele code van Labo 3 van het opleidingsonderdeel ``Machine Learning Operations''. Hierna werden alle bestaande onderdelen omgezet naar het gekozen framework. Dit werd gedaan aan de hand van de documentatie van het framework. Bij sommige frameworks moest de code meer aangepast worden dan bij andere frameworks, maar de flow moest nog steeds hetzelfde blijven.
De flow van de pipeline is als volgt: eerst worden alle afbeeldingen gedownload en in de juiste folder geplaatst, hierna zal de preprocess functie alle foto's bewerken zodat ze het juiste formaat hebben. Vervolgens wordt met alle afbeeldingen een convolutional neural network (CNN) model getraind. Als laatste wordt dit model geëvalueerd om de prestatie ervan te meten.
Elk framework toont de flow van de pipeline op zijn eigen manier en dit zal in de Proof of Concept worden uitgelegd, samen met de werking van het omzetten van de originele code.
\section{Doel van de Proof of Concept}
De Proof of Concept heeft als doel om de haalbaarheid te onderzoeken van het uitvoeren van machine learning pipelines in een lokale omgeving. Hiervoor worden verschillende frameworks of libraries gebruikt, die moeten voldoen aan de requirementsanalyse. De Proof of Concept zal drie frameworks uitvoeren die dezelfde taak uitvoeren zoals in Labo 3. Hierna wordt er gekeken of het optimaal is om met het gekozen framework een Machine Learning pipeline uit te voeren.
\section{Conclusie}
Ten slotte zullen alle bevindingen die voortvloeien uit de Proof of Concept worden getoetst aan de oorspronkelijke vereisten die zijn vastgesteld in Sectie \ref{s:Requirementsanalyse}. Hierbij wordt beoordeeld of aan alle vereisten is voldaan en of de onderzoeksvraag beantwoord kan worden.
%% TODO: In dit hoofstuk geef je een korte toelichting over hoe je te werk bent
%% gegaan. Verdeel je onderzoek in grote fasen, en licht in elke fase toe wat
%% de doelstelling was, welke deliverables daar uit gekomen zijn, en welke
%% onderzoeksmethoden je daarbij toegepast hebt. Verantwoord waarom je
%% op deze manier te werk gegaan bent.
%% 
%% Voorbeelden van zulke fasen zijn: literatuurstudie, opstellen van een
%% requirements-analyse, opstellen long-list (bij vergelijkende studie),
%% selectie van geschikte tools (bij vergelijkende studie, "short-list"),
%% opzetten testopstelling/PoC, uitvoeren testen en verzamelen
%% van resultaten, analyse van resultaten, ...
%%
%% !!!!! LET OP !!!!!
%%
%% Het is uitdrukkelijk NIET de bedoeling dat je het grootste deel van de corpus
%% van je bachelorproef in dit hoofstuk verwerkt! Dit hoofdstuk is eerder een
%% kort overzicht van je plan van aanpak.
%%
%% Maak voor elke fase (behalve het literatuuronderzoek) een NIEUW HOOFDSTUK aan
%% en geef het een gepaste titel.

