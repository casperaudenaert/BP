%%=============================================================================
%% Methodologie
%%=============================================================================

\chapter{\IfLanguageName{dutch}{Methodologie}{Methodology}}%
\label{ch:methodologie}

% TODO: geen sectie zonder inleiding ervoor schrijven + geen subsectie zonder section
% TODO: hang deze verschillende fasen aan elkaar met een verhaal, nu stan ze redelijk los van elkaar

Voor de start van dit onderzoek is het noodzakelijk om een basiskennis te verwerven over de verschillende concepten en de verschillende Machine Learning frameworks die aan bod zullen komen. De volgende hoodstukken beschrijven de fasen die werden uitgevoerd voor dit onderzoek.

\section*{Literatuurstudie}
Deze literatuurstudie leidt tot een oplijsting van verschillende frameworks en libraries die gebruikt zullen worden in dit onderzoek. Alsook de uitleg van verschillende termen en de werking van bepaalde technologieën.
De frameworks en libraries worden aan de hand van een internet studie en de requirementsanalyse gekozen. Deze worden dan nauwkeurig beschreven in de stand van zaken. De beschrijving bevat wat elke framework is met daarop volgend de voor en nadelen van het framework.
Dit zal aan de hand van de officiële documentatie en verschillende onderzoekstudies geschreven worden.
De literatuurstudie zorgt voor een long list waaruit enkele frameworks en libararies gekozen gaan worden om een Proof of Concept op te stellen.
Op basis van voorgedefinieerde functionele en niet functionele vereisten zal vervolgens een geschikte framework of libary worden gekozen.
\section*{voorbereiding proof of concept}
\section*{Proof of Concept}
\section*{Machine Learning pipeline}
\section*{Conclusie}
%% TODO: In dit hoofstuk geef je een korte toelichting over hoe je te werk bent
%% gegaan. Verdeel je onderzoek in grote fasen, en licht in elke fase toe wat
%% de doelstelling was, welke deliverables daar uit gekomen zijn, en welke
%% onderzoeksmethoden je daarbij toegepast hebt. Verantwoord waarom je
%% op deze manier te werk gegaan bent.
%% 
%% Voorbeelden van zulke fasen zijn: literatuurstudie, opstellen van een
%% requirements-analyse, opstellen long-list (bij vergelijkende studie),
%% selectie van geschikte tools (bij vergelijkende studie, "short-list"),
%% opzetten testopstelling/PoC, uitvoeren testen en verzamelen
%% van resultaten, analyse van resultaten, ...
%%
%% !!!!! LET OP !!!!!
%%
%% Het is uitdrukkelijk NIET de bedoeling dat je het grootste deel van de corpus
%% van je bachelorproef in dit hoofstuk verwerkt! Dit hoofdstuk is eerder een
%% kort overzicht van je plan van aanpak.
%%
%% Maak voor elke fase (behalve het literatuuronderzoek) een NIEUW HOOFDSTUK aan
%% en geef het een gepaste titel.

