%%=============================================================================
%% Methodologie
%%=============================================================================

\chapter{\IfLanguageName{dutch}{Methodologie}{Methodology}}%
\label{ch:methodologie}

% TODO: geen sectie zonder inleiding ervoor schrijven + geen subsectie zonder section
% TODO: hang deze verschillende fasen aan elkaar met een verhaal, nu stan ze redelijk los van elkaar

Voor de start van dit onderzoek is het noodzakelijk om een basiskennis te verwerven over de verschillende concepten en de verschillende Machine Learning frameworks die aan bod zullen komen. De volgende hoodstukken beschrijven de fasen die werden uitgevoerd voor dit onderzoek.

\section{Literatuurstudie}
Deze literatuurstudie leidt tot een oplijsting van verschillende frameworks en libraries die gebruikt zullen worden in dit onderzoek. Alsook de uitleg van verschillende termen en de werking van bepaalde technologieën.
De frameworks en libraries worden aan de hand van een internet studie en de requirementsanalyse gekozen. Deze worden dan nauwkeurig beschreven in de stand van zaken. De beschrijving bevat wat elke framework is met daarop volgend de voor en nadelen van het framework.
Dit zal aan de hand van de officiële documentatie en verschillende onderzoekstudies geschreven worden.
De literatuurstudie zorgt voor een long list waaruit enkele frameworks en libararies gekozen gaan worden om een Proof of Concept op te stellen.
Op basis van voorgedefinieerde functionele en niet functionele vereisten zal vervolgens een geschikte framework of libary worden gekozen.
\section{voorbereiding proof of concept}
Vooraleer dat er gestart kan worden aan Proof of Concept moet er een goed begrip gekregen worden van de gekozen frameworks en libraries. 
\subsection{Kennis van frameworks en libraries}
Een goed begrip van de gekozen frameworks en libraries is belangrijk voor een goede uitwerking van de Proof of Concept. Hieronder volgt een lijst met de belangerijkste aspecten:
\begin{itemize}
  \item Documentatie: De documentatie van het gekozen framework of library kan meer uitleg geven over de werking en de functionaliteiten ervan.
  \item Syntax: Elke framework of library heeft zijn eigen syntax, het is belangerijk om deze syntax goed te verstaan en te kunnen gebruiken.
  \item Functionaliteiten: Voor elk framework worden de functionaliteiten opgelijst.
\end{itemize}
\section{Proof of Concept}
Vervolgens werd er verschillende Proof of Concepts opgesteld om te onderzoeken of de onderzoeksvraag en deelvragen uit de inleiding beantwoord kunnen worden.

De Proof of Concepts zullen bestaan uit de code van Labo 3 uit het opleidingsonderdeel ``Machine Learning Operations'' die word omgezet naar de gekozen frameworks.
\section{Machine Learning pipeline}
Er werd gekeken naar de orginele code van Labo 3 van het opleidingsonderdeel ``Machine Learning Operations'', hierna werden alle bestaande onderdelen omgezet naar het gekozen framework. Dit werd aan de hand van de documentatie van het framework gedaan.
Bij sommige frameworks moest de code meer aangepast worden dan andere frameworks maar de flow moet nog steeds zijn het zelfde zijn.
De flow van de pipeline is als volgend, eerst worden alle afbeelding gedownload en in de juiste folder geplaats. Hierna zal het alle foto's bewerken zodat het een juist formaat is. Vervolgens is er met alle afbeelden een CCN model getraint geweest. Als laatste werd dit model gevalueert om de prestatie ervan te meten.
Elke framework toont de flow van de pipeline op zijn eigen manier en dit zal in de Proof of Concept worden uitgelegd, samen met de werking van het omzetten van de orginele code.
\section{Doel van de proof of concept}
De Proof of Concept heeft als doel om de haalbaarheid te onderzoeken van het uitvoeren van machine learning pipelines in een lokale omgeving. Hiervoor worden verschillende frameworks of libraries  gebruikt, deze moeten voldoen aan de requirementsanalyse.
Het is van cruciaal belang dat het gebruik van een framework of library de prestatie van de pipeline geen significante negatieve invloed heeft. De Proof of Concept zal 3 frameworks uitvoeren, deze zal dan de zelfde taak uitvoeren zoals in Labo 3. Hierna word er gekeken of het wel optimaal is om met het gekozen framework een Machine Learning pipeline uit te voeren.
\section{Conclusie}
Ten slotte zullen alle bevindingen die voortvloeien uit de Proof of Concept worden getoets aan de oorspronkelijke vereisten die zijn vastgesteld in Sectie . Hierbij
wordt beoordeeld of aan alle vereisten is voldaan en of de onderzoeksvraag beantwoord kan worden.
%% TODO: In dit hoofstuk geef je een korte toelichting over hoe je te werk bent
%% gegaan. Verdeel je onderzoek in grote fasen, en licht in elke fase toe wat
%% de doelstelling was, welke deliverables daar uit gekomen zijn, en welke
%% onderzoeksmethoden je daarbij toegepast hebt. Verantwoord waarom je
%% op deze manier te werk gegaan bent.
%% 
%% Voorbeelden van zulke fasen zijn: literatuurstudie, opstellen van een
%% requirements-analyse, opstellen long-list (bij vergelijkende studie),
%% selectie van geschikte tools (bij vergelijkende studie, "short-list"),
%% opzetten testopstelling/PoC, uitvoeren testen en verzamelen
%% van resultaten, analyse van resultaten, ...
%%
%% !!!!! LET OP !!!!!
%%
%% Het is uitdrukkelijk NIET de bedoeling dat je het grootste deel van de corpus
%% van je bachelorproef in dit hoofstuk verwerkt! Dit hoofdstuk is eerder een
%% kort overzicht van je plan van aanpak.
%%
%% Maak voor elke fase (behalve het literatuuronderzoek) een NIEUW HOOFDSTUK aan
%% en geef het een gepaste titel.

