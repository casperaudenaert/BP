%%=============================================================================
%% Samenvatting
%%=============================================================================

% TODO: De "abstract" of samenvatting is een kernachtige (~ 1 blz. voor een
% thesis) synthese van het document.
%
% Een goede abstract biedt een kernachtig antwoord op volgende vragen:
%
% 1. Waarover gaat de bachelorproef?
% 2. Waarom heb je er over geschreven?
% 3. Hoe heb je het onderzoek uitgevoerd?
% 4. Wat waren de resultaten? Wat blijkt uit je onderzoek?
% 5. Wat betekenen je resultaten? Wat is de relevantie voor het werkveld?
%
% Daarom bestaat een abstract uit volgende componenten:
%
% - inleiding + kaderen thema
% - probleemstelling
% - (centrale) onderzoeksvraag
% - onderzoeksdoelstelling
% - methodologie
% - resultaten (beperk tot de belangrijkste, relevant voor de onderzoeksvraag)
% - conclusies, aanbevelingen, beperkingen
%
% LET OP! Een samenvatting is GEEN voorwoord!

%%---------- Nederlandse samenvatting -----------------------------------------
%
% TODO: Als je je bachelorproef in het Engels schrijft, moet je eerst een
% Nederlandse samenvatting invoegen. Haal daarvoor onderstaande code uit
% commentaar.
% Wie zijn bachelorproef in het Nederlands schrijft, kan dit negeren, de inhoud
% wordt niet in het document ingevoegd.

\IfLanguageName{english}{%
\selectlanguage{dutch}
\chapter*{Samenvatting}
\lipsum[1-4]
\selectlanguage{english}
}{}

%%---------- Samenvatting -----------------------------------------------------
% De samenvatting in de hoofdtaal van het document

\chapter*{\IfLanguageName{dutch}{Samenvatting}{Abstract}}

Dit onderzoeksvoorstel richt zich op het verkennen van de mogelijkheden voor het lokaal uitvoeren van machine learning pipelines, aan de hand van verschillende frameworks.
Binnen het keuzepakket ``AI \& Data Engineering'' en het bijbehorende opleidingsonderdeel ``Machine Learning Operations'' wordt er momenteel gebruik gemaakt van Azure ML pipelines voor het uitvoeren van machine learning pipelines in de Azure Cloud.
Het gebruik van Azure ML pipelines is echter niet gratis voor studenten die niet voldoende krediet hebben op het platform. Hierdoor richt dit onderzoek zich op het lokaal uitvoeren van machine learning pipelines.
De basis van Azure ML pipelines is Kubeflow. Voor het lokaal draaien van Kubeflow zijn er echter aanzienlijke uitdagingen, waardoor het noodzakelijk is om een aangepaste Proof of Concept (PoC) op te stellen.
Het lokaal uitvoeren van machine learning pipelines kan ook problemen met zich meebrengen, zoals de noodzaak van aanzienlijke rekenkracht die mogelijk niet beschikbaar is op computers van studenten. Als gevolg daarvan hebben bepaalde frameworks een beperktere versie die niet zo krachtig is als hun cloud-varianten, maar toch lokaal machine learning pipelines kan uitvoeren.
De eerste fase van deze bachelorproef omvat een uitgebreide literatuurstudie waarin onderzocht wordt welke frameworks bestaan voor het lokaal draaien van machine learning pipelines, hoe deze frameworks functioneren, en ook de compatibiliteit en de diverse functies ervan.
Het tweede deel van dit onderzoek richt zich op het opzetten van een Proof of Concept voor de geselecteerde tool(s), mogelijks vertrekkende van publiek beschikbare manifest bestanden. Hierbij wordt specifiek aandacht besteed aan het gemak van onderhoud voor de betrokken lectoren.
Het verwachte resultaat van dit onderzoek is een diepgaand inzicht in de geschiktheid van verschillende frameworks voor het lokaal uitvoeren van machine learning pipelines, met praktische toepasbaarheid in het onderwijs en het bedrijfsleven.
