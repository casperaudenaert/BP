%%=============================================================================
%% Voorwoord
%%=============================================================================

\chapter*{\IfLanguageName{dutch}{Woord vooraf}{Preface}}%
\label{ch:voorwoord}
Deze bachelorproef markeert het einde van de Bachelor in de Toegepaste Informatica aan de Hogeschool Gent, en is gericht op de specialisatie Data Engineering \& AI. Aangezien ik veel in contact kom met data en Machine Learning, leek het mij een goede keuze om dit onderwerp te kiezen. Het leek mij vooral interessant om verschillende frameworks te onderzoeken en om mijn eigen oplossing te kunnen presenteren.

Ik heb al gewerkt met een cloudomgeving voor dit onderwerp, maar lokaal werken is een volledig andere uitdaging. Dit motiveerde mij dan ook om verder te werken en een oplossing te vinden. De onderzoeksvraag werd oorspronkelijk gesteld door Thomas Aelbrecht; toen ik contact met hem opnam, hebben we samen naar het onderwerp gekeken.

Na deze bespreking was ik helemaal overtuigd en wilde ik dit onderwerp zeker uitwerken. Gedurende dit onderzoek kon ik altijd rekenen op de hulp van mijn promotor Stijn Lievens en mijn copromotor Thomas Aelbrecht. Ze hebben mij gedurende de hele bachelorproef goed opgevolgd en geholpen wanneer er problemen waren. Mijn oprechte dank gaat uit naar hen voor de tijd en moeite die ze hebben geïnvesteerd om mijn bachelorproef tot een goed einde te brengen.

% TODO: bedank ook je ouders, vrienden, partner... die je gesteund hebben gedurende je opleiding.

% Deze bachlorproef geeft het einde aan van de Bachelor in de Toegepaste Informatica aan de Hogeschool Gent. Dit is gericht op specialisatie Data Engineering \& AI. Aangezien dat ik veel in contact kom met data en Machine Learning leek het mij een goede keus om dit onderwerp te kiezen. Het leek mij vooral interessant dat ik gaan onderzoeken naar verschillende frameworks en dat ik mijn eigen oplossing kan voortonen.
% Ik heb al gewert met een cloud omgeving voor dit onderwerp maar lokaal is een volledig andere uitdaging en dat gaf mij ook motivatie om verder te werken en een oplossing te vinden. De onderzoeksvraag werd ogineel gesteld door meneer Aelbrecht, wanneer ik hem contacteerde heb ik eens met hem samen naar het onderwerp gekeken.
% Na deze besprekening was ik helemaal verkocht en wou ik zeker dit onderwerp uitwerken. Gedurende dit onderzoek kon ik altijd op de hulp rekenen van mijn promotor Stijn Lievens en mijn copromotor Thomas Aelbrecht. Ze hebbem mij gedurende heel de bachelorproef goed opgevolgd en geholpen wanneer er problemen waren. Mijn oprechte dank gaat uit naar hun voor de tijd en moeite die ze hebben geïnvesteerd om mijn bachelorproef tot een goede einde te brengen.
%% TODO:
%% Het voorwoord is het enige deel van de bachelorproef waar je vanuit je
%% eigen standpunt (``ik-vorm'') mag schrijven. Je kan hier bv. motiveren
%% waarom jij het onderwerp wil bespreken.
%% Vergeet ook niet te bedanken wie je geholpen/gesteund/... heeft
