%%=============================================================================
%% Conclusie
%%=============================================================================

\chapter{Conclusie}%
\label{ch:conclusie}

% TODO: Trek een duidelijke conclusie, in de vorm van een antwoord op de
% onderzoeksvra(a)g(en). Wat was jouw bijdrage aan het onderzoeksdomein en
% hoe biedt dit meerwaarde aan het vakgebied/doelgroep? 
% Reflecteer kritisch over het resultaat. In Engelse teksten wordt deze sectie
% ``Discussion'' genoemd. Had je deze uitkomst verwacht? Zijn er zaken die nog
% niet duidelijk zijn?
% Heeft het onderzoek geleid tot nieuwe vragen die uitnodigen tot verder 
%onderzoek?


% TODO: zorg ervoor dat je alle onderzoeksvragen beantwoord
% TODO: zorg voor een minder lange tekst, verdeel hem in meerdere paragrafen

% Het probleem dat zich voordoet bij de huidige uitvoering van Machine Learning pipelines is dat deze bijna altijd in de cloud worden uitgevoerd. De cloud biedt veel opties en tarieven, waardoor dit voor studenten niet haalbaar is vanwege de kosten.
% Daarom is in dit onderzoek gekeken naar de mogelijkheid om Machine Learning pipelines lokaal uit te voeren om dit probleem aan te pakken. Verschillende frameworks die lokaal Machine Learning pipelines kunnen uitvoeren, worden vergeleken. De voor- en nadelen van elk framework worden geanalyseerd, zodat het beste framework kan worden gekozen.
% Na het analyseren van deze frameworks is de keuze gevallen op het Prefect framework. Prefect heeft een gebruiksvriendelijke interface, is schaalbaar met de cloud en biedt veel functies voor het beheren en ontwikkelen van Machine Learning pipelines. Dit framework zorgt er ook voor dat de code in dezelfde structuur als een normale Machine Learning pipeline kan blijven, in vergelijking met andere frameworks waarbij de code nog moet worden aangepast op basis van het doel van het project.
% Alle frameworks zijn in staat om een Machine Learning Pipeline uit te voeren, maar velen zijn beperkt tot tekstgebaseerde Machine Learning pipelines. Dit onderzoek maakte gebruik van een image classification en alleen Prefect kon dit zonder veel aanpassingen uitvoeren als een Machine Learning pipeline. De andere frameworks hadden problemen met het doorgeven van variabelen, waardoor het langer duurde om een normale Machine Learning pipeline over te zetten naar het gekozen framework.
% Prefect biedt ook meer functionaliteiten dan andere frameworks, zoals monitoring, herstarten van mislukte flows, plannen van uitvoeringen en een goed overzicht van alle flows. Het gebruik van Prefect biedt een goede basis voor het lokaal uitvoeren van Machine Learning pipelines.
% Hierdoor kan Prefect worden beschouwd als een waardevol instrument voor data scientists en data engineers die op zoek zijn naar een betrouwbare oplossing voor het beheren van hun Machine Learning-workflows in een lokale omgeving.

De Proof of Concepts hebben inzicht geboden in de efficiëntie en bruikbaarheid van de drie onderzochte frameworks - Prefect, ZenML en Dagster - voor het lokaal uitvoeren van machine learning pipelines. Deze analyse biedt een basis voor aanbevelingen over welk framework het meest geschikt is voor gebruik in het opleidingsonderdeel Machine Learning Operations.\\

Prefect is een krachtig framework dat gebruikmaakt van een intuïtieve gebruikersinterface en een uitgebreide set functies biedt voor het beheren en uitvoeren van workflows. Het biedt een eenvoudige installatie en configuratie, waardoor gebruikers snel aan de slag kunnen. Prefect integreert naadloos met MLFlow voor experiment tracking, waardoor het gemakkelijk is om de resultaten van machine learning modellen te volgen en te beheren.\\

ZenML biedt een gestructureerde aanpak voor het ontwikkelen en uitvoeren van machine learning pipelines. Het framework maakt gebruik van decorators om Py\-thon-functies om te zetten naar ZenML-stappen, waardoor het ontwikkelingsproces vereenvoudigd wordt. Hoewel ZenML een gebruiksvriendelijke interface biedt, zijn er enkele uitdagingen geïdentificeerd, zoals beperkte ondersteuning voor bepaalde datatypes en enkele compatibiliteitsproblemen met oudere library-versies.\\

Dagster richt zich op de \textit{data orchestration} en biedt een krachtig platform voor het bouwen en uitvoeren van data pipelines. Het maakt gebruik van een concept genaamd \textit{assets} om taken te definiëren en te orchestreren. Hoewel Dagster een robuust framework is, vereist het een iets complexere configuratie in vergelijking met de andere twee frameworks.\\

Gezien de vergelijking van de frameworks en de resultaten van de Proof of Concepts, wordt Prefect aanbevolen voor gebruik in het opleidingsonderdeel Machine Learning Operations. Prefect combineert een gebruiksvriendelijke interface met krachtige functies voor workflowbeheer en integratie met MLFlow voor experiment tracking. Het biedt een ideale balans tussen eenvoud, efficiëntie en functionaliteit, waardoor het geschikt is voor zowel educatieve als professionele toepassingen.

Met Prefect kunnen studenten effectief machine learning pipelines ontwikkelen en beheren, terwijl ze tegelijkertijd vertrouwd raken met industriestandaard tools en workflows. Dit stelt hen in staat om hun vaardigheden op het gebied van machine learning operations verder te ontwikkelen en voor te bereiden op toekomstige uitdagingen in het werkveld.


