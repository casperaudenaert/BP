\section{Inleiding}
Het Prefect framework is een open-source tool voor het orkestreren van machine learning pipelines. Het biedt een platformonafhankelijke oplossing om pipelines te bouwen, te beheren en te monitoren, zowel lokaal als in de cloud. In dit literatuurstuk bespreken we de werking van Prefect, de voordelen en nadelen van het gebruik ervan, en de hardware- en softwarevereisten voor het lokaal uitvoeren van pipelines met Prefect.

\section{Hoe werkt Prefect?}
Prefect pipelines bestaan uit taken, die afzonderlijke stappen in de pipeline vertegenwoordigen. Taken kunnen worden gedefinieerd in Python-code of met behulp van Prefect's ingebouwde taakbibliotheek. De taken worden geordend in een gerichte acyclische graaf (DAG), die de volgorde van de uitvoering bepaalt.

Prefect biedt twee modi voor het uitvoeren van pipelines:

\begin{itemize}
    \item Lokaal: Pipelines kunnen lokaal worden uitgevoerd op een machine met Python 3.7 of hoger.
    \item Cloud: Prefect Cloud is een SaaS-platform dat het beheren en uitvoeren van pipelines in de cloud vereenvoudigt.
\end{itemize}

\section{De voordelen van Prefect}
\begin{itemize}
    \item Eenvoudig te gebruiken: Prefect biedt een intuïtieve interface voor het definiëren en beheren van pipelines.
    \item Flexibel: Prefect pipelines kunnen worden uitgevoerd op verschillende platforms, inclusief lokaal, cloud en Kubernetes.
    \item Schaalbaar: Prefect pipelines kunnen worden geschaald om te voldoen aan de behoeften van uw project.
    \item Reproduceerbaar: Prefect pipelines garanderen reproduceerbare resultaten door de uitvoering te herleiden tot gedefinieerde taken en parameters.
    \item Moniteerbaar: Prefect biedt uitgebreide monitoringmogelijkheden om de voortgang en prestaties van uw pipelines te volgen.
\end{itemize}

\section{De nadelen van Prefect}
\begin{itemize}
    \item Relatief nieuw: Prefect is een relatief nieuw framework en heeft nog niet de uitgebreide community en documentatie van andere frameworks.
    \item Ondersteuning voor cloudproviders: Prefect Cloud biedt momenteel alleen ondersteuning voor AWS en Azure.
    \item Complexiteit: Voor complexe pipelines kan het definiëren van de DAG en het beheren van de afhankelijkheden tussen taken een uitdaging zijn.
\end{itemize}

\section{Hardware- en softwarevereisten voor lokale uitvoering}
\subsection{Hardware}
De hardwarevereisten voor het lokaal uitvoeren van Prefect pipelines zijn afhankelijk van de complexiteit van de pipeline en de hoeveelheid data die wordt verwerkt. In het algemeen is een machine met een moderne processor, voldoende RAM en opslagruimte vereist.
\subsection{Software}
De volgende software is vereist voor het lokaal uitvoeren van Prefect pipelines:
\begin{itemize}
    \item Python 3.7 of hoger
    \item Pipenv (optioneel)
    \item Docker (optioneel)
\end{itemize}

\section{Codevoorbeeld}
\subsection{Python}
\begin{minted}[frame=lines,linenos]{python}
from prefect import task, Flow

@task
def extract_data():
    # ...

@task
def transform_data():
    # ...

@task
def load_data():
    # ...

with Flow("my-pipeline") as flow:
    extract_data()
    transform_data()
    load_data()

flow.run()
\end{minted}

\section{Conclusie}
Het Prefect framework is een krachtige tool voor het orkestreren van machine learning pipelines. Het biedt een gebruiksvriendelijke interface, flexibiliteit en schaalbaarheid. De nadelen van Prefect zijn de relatief nieuwe status, de beperkte ondersteuning voor cloudproviders en de potentiële complexiteit van het definiëren van complexe pipelines.