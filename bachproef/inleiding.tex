%%=============================================================================
%% Inleiding
%%=============================================================================

\chapter{\IfLanguageName{dutch}{Inleiding}{Introduction}}%
\label{ch:inleiding}

De inleiding moet de lezer net genoeg informatie verschaffen om het onderwerp te begrijpen en in te zien waarom de onderzoeksvraag de moeite waard is om te onderzoeken. In de inleiding ga je literatuurverwijzingen beperken, zodat de tekst vlot leesbaar blijft. Je kan de inleiding verder onderverdelen in secties als dit de tekst verduidelijkt. Zaken die aan bod kunnen komen in de inleiding~\autocite{Pollefliet2011}:

\begin{itemize}
  \item context, achtergrond
  \item afbakenen van het onderwerp
  \item verantwoording van het onderwerp, methodologie
  \item probleemstelling
  \item onderzoeksdoelstelling
  \item onderzoeksvraag
  \item \ldots
\end{itemize}

\section{\IfLanguageName{dutch}{Probleemstelling}{Problem Statement}}%
\label{sec:probleemstelling}

Om gebruik te kunnen maken van de Azure servers, hebben gebruikers credits nodig. Tijdens de eerder genoemde opdracht waren er studenten die voor aanvang van of tijdens de opdracht geen credits meer hadden. Dit leidde ertoe dat deze studenten hun resultaten niet konden presenteren of zelfs niet konden beginnen aan de opdracht. Daarom wordt in dit onderzoeksvoorstel gekeken naar het opzetten van een lokale installatie van een Machine Learning pipeline.


\section{\IfLanguageName{dutch}{Onderzoeksvraag}{Research question}}%
\label{sec:onderzoeksvraag}

Het onderzoek richt zich op het verkennen van mogelijkheden voor het lokaal uitvoeren van machine learning pipelines met behulp van verschillende frameworks. De centrale onderzoeksvraag luidt: ``Hoe kunnen machine learning pipelines lokaal worden uitgevoerd met behulp van diverse machine learning frameworks?''

Deze centrale onderzoeksvraag kan verder worden gespecificeerd aan de hand van de volgende deelvragen:
\begin{itemize}
  \item Wat zijn de belangrijkste uitdagingen bij het lokaal uitvoeren van machine learning pipelines?
  \item Welke specifieke problemen kunnen optreden bij het lokaal uitvoeren van machine learning pipelines?
  \item Hoe vertaalt de lokale uitvoering van een machine learning pipeline zich naar de cloud?
  \item Wat zijn de hardware- en softwarevereisten voor het lokaal uitvoeren van een machine learning pipeline?
  \item Welke frameworks en tools kunnen worden gebruikt om machine learning pipelines lokaal uit te voeren, en wat zijn hun respectieve kenmerken en voordelen?
\end{itemize}

Door deze deelvragen te beantwoorden, wordt een diepgaand inzicht verkregen in de geschiktheid van verschillende frameworks voor het lokaal uitvoeren van machine learning pipelines, met praktische toepasbaarheid in zowel het onderwijs als het bedrijfsleven.

\section{\IfLanguageName{dutch}{Onderzoeksdoelstelling}{Research objective}}%
\label{sec:onderzoeksdoelstelling}

Wat is het beoogde resultaat van je bachelorproef? Wat zijn de criteria voor succes? Beschrijf die zo concreet mogelijk. Gaat het bv.\ om een proof-of-concept, een prototype, een verslag met aanbevelingen, een vergelijkende studie, enz.

\section{\IfLanguageName{dutch}{Opzet van deze bachelorproef}{Structure of this bachelor thesis}}%
\label{sec:opzet-bachelorproef}

% Het is gebruikelijk aan het einde van de inleiding een overzicht te
% geven van de opbouw van de rest van de tekst. Deze sectie bevat al een aanzet
% die je kan aanvullen/aanpassen in functie van je eigen tekst.

De rest van deze bachelorproef is als volgt opgebouwd:

In Hoofdstuk~\ref{ch:stand-van-zaken} wordt een overzicht gegeven van de stand van zaken binnen het onderzoeksdomein, op basis van een literatuurstudie.

In Hoofdstuk~\ref{ch:methodologie} wordt de methodologie toegelicht en worden de gebruikte onderzoekstechnieken besproken om een antwoord te kunnen formuleren op de onderzoeksvragen.

% TODO: Vul hier aan voor je eigen hoofstukken, één of twee zinnen per hoofdstuk
In Hoofdstuk~\ref{ch:PoC} word het Proof of Concept toegelicht en beschreven hoe dat dit werd uitgewerkt.

In Hoofdstuk~\ref{ch:conclusie}, tenslotte, wordt de conclusie gegeven en een antwoord geformuleerd op de onderzoeksvragen. Daarbij wordt ook een aanzet gegeven voor toekomstig onderzoek binnen dit domein.