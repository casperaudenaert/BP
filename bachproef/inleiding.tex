%%=============================================================================
%% Inleiding
%%=============================================================================

\chapter{\IfLanguageName{dutch}{Inleiding}{Introduction}}%
\label{ch:inleiding}

De inleiding moet de lezer net genoeg informatie verschaffen om het onderwerp te begrijpen en in te zien waarom de onderzoeksvraag de moeite waard is om te onderzoeken. In de inleiding ga je literatuurverwijzingen beperken, zodat de tekst vlot leesbaar blijft. Je kan de inleiding verder onderverdelen in secties als dit de tekst verduidelijkt. Zaken die aan bod kunnen komen in de inleiding~\autocite{Pollefliet2011}:

\begin{itemize}
  \item context, achtergrond
  \item afbakenen van het onderwerp
  \item verantwoording van het onderwerp, methodologie
  \item probleemstelling
  \item onderzoeksdoelstelling
  \item onderzoeksvraag
  \item \ldots
\end{itemize}

\section{\IfLanguageName{dutch}{Probleemstelling}{Problem Statement}}%
\label{sec:probleemstelling}

% Deze bachelorproef richt zich op twee belangrijke doelgroepen: IT-professionals betrokken bij Machine Learning Operations, waaronder data scientists, machine learning engineers, DevOps engineers, en IT-managers, alsook studenten in de opleiding Toegepaste Informatica, met een focus op AI & Data Engineering. Het identificeert een aantal problemen met de huidige praktijk van het uitvoeren van Machine Learning pipelines, met name de afhankelijkheid van cloudplatforms zoals Azure ML, wat niet altijd haalbaar of wenselijk is vanwege kosten, beveiliging en betrouwbaarheidsoverwegingen.

% Het onderzoeksproject stelt een centrale onderzoeksvraag: ``Hoe kan een Machine Learning pipeline lokaal worden uitgevoerd met behulp van een Machine Learning framework, met oog op de noden van de doelgroep?'' Om deze vraag te beantwoorden, worden verschillende deelvragen aangepakt, zoals de belangrijkste uitdagingen bij het lokaliseren van Machine Learning pipelines, specifieke problemen die kunnen optreden bij het lokaal uitvoeren ervan, en hoe deze lokale uitvoering zich vertaalt naar de cloud.

% De meerwaarde van Deze bachelorproef ligt in het bieden van concrete richtlijnen, best practices en aanbevelingen voor het lokaal uitvoeren van ML pipelines, waardoor de efficiëntie en productiviteit van de doelgroep worden verhoogd. Het zal ook bijdragen aan een verbeterd begrip en expertise in Machine Learning Operations, zowel voor IT-professionals als studenten. Door te focussen op de behoeften van de doelgroep, streeft deze bachelorproef naar maximale impact en een blijvende bijdrage aan het kennisdomein van Machine Learning Operations.

\section{\IfLanguageName{dutch}{Onderzoeksvraag}{Research question}}%
\label{sec:onderzoeksvraag}

% Het onderzoek richt zich op het verkennen van mogelijkheden voor het lokaal uitvoeren van machine learning pipelines met behulp van verschillende frameworks. De centrale onderzoeksvraag luidt: ``Hoe kunnen machine learning pipelines lokaal worden uitgevoerd met behulp van diverse machine learning frameworks?''

% Deze centrale onderzoeksvraag kan verder worden gespecificeerd aan de hand van de volgende deelvragen:
% \begin{itemize}
%   \item Wat zijn de belangrijkste uitdagingen bij het lokaal uitvoeren van machine learning pipelines?
%   \item Welke specifieke problemen kunnen optreden bij het lokaal uitvoeren van machine learning pipelines?
%   \item Hoe vertaalt de lokale uitvoering van een machine learning pipeline zich naar de cloud?
%   \item Wat zijn de hardware- en softwarevereisten voor het lokaal uitvoeren van een machine learning pipeline?
%   \item Welke frameworks en tools kunnen worden gebruikt om machine learning pipelines lokaal uit te voeren, en wat zijn hun respectieve kenmerken en voordelen?
% \end{itemize}

% Door deze deelvragen te beantwoorden, wordt een diepgaand inzicht verkregen in de geschiktheid van verschillende frameworks voor het lokaal uitvoeren van machine learning pipelines, met praktische toepasbaarheid in zowel het onderwijs als het bedrijfsleven.

\section{\IfLanguageName{dutch}{Onderzoeksdoelstelling}{Research objective}}%
\label{sec:onderzoeksdoelstelling}

Wat is het beoogde resultaat van je bachelorproef? Wat zijn de criteria voor succes? Beschrijf die zo concreet mogelijk. Gaat het bv.\ om een proof-of-concept, een prototype, een verslag met aanbevelingen, een vergelijkende studie, enz.

\section{\IfLanguageName{dutch}{Opzet van deze bachelorproef}{Structure of this bachelor thesis}}%
\label{sec:opzet-bachelorproef}

% Het is gebruikelijk aan het einde van de inleiding een overzicht te
% geven van de opbouw van de rest van de tekst. Deze sectie bevat al een aanzet
% die je kan aanvullen/aanpassen in functie van je eigen tekst.

De rest van deze bachelorproef is als volgt opgebouwd:

In Hoofdstuk~\ref{ch:stand-van-zaken} wordt een overzicht gegeven van de stand van zaken binnen het onderzoeksdomein, op basis van een literatuurstudie.

In Hoofdstuk~\ref{ch:methodologie} wordt de methodologie toegelicht en worden de gebruikte onderzoekstechnieken besproken om een antwoord te kunnen formuleren op de onderzoeksvragen.

% TODO: Vul hier aan voor je eigen hoofstukken, één of twee zinnen per hoofdstuk
In Hoofdstuk~\ref{ch:PoC} word het Proof of Concept toegelicht en beschreven hoe dat dit werd uitgewerkt.

In Hoofdstuk~\ref{ch:conclusie}, tenslotte, wordt de conclusie gegeven en een antwoord geformuleerd op de onderzoeksvragen. Daarbij wordt ook een aanzet gegeven voor toekomstig onderzoek binnen dit domein.