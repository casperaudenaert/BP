%---------- Inleiding ---------------------------------------------------------

\section{Introductie}%
\label{sec:introductie}

Het toenemende belang van Artificial Intelligence (AI), Machine Learning (ML) en Deep Learning heeft geleid tot een groeiende vraag naar effectieve methoden om machine learning modellen op een schaalbare en efficiënte manier te implementeren en te beheren.
Binnen het vakgebied "AI \& Data Engineering" is er een cursus Machine Learning Operations, waar studenten leren van het opzetten van machine learning workspaces tot het monitoren van machine learning operations.
Dit onderzoek richt zich op een specifieke uitdaging binnen machine learning operations: Het lokaal draaien van machine learning pipelines.

In deze context word er gebruik gemaakt van Azure ML pipelines, de onderliggende framework hiervan is Kubeflow. Kubeflow is een open-source platform voor machine learning en MLOps op Kubernetes. Kubeflow zal centraal staan in dit onderzoek. %TODO bron toevoegen van Kubeflow
Er zal ook onderoek gedaan worden naar alternatieven en die worden dan vergeleken met elkaar om eventueel een tweede PoC op te stellen.
De focus ligt op het lokaal draaien van Kubeflow, dit proces is uitdagend door do onduidelijke documentatie en de migratieproblemen van versie 1 naar versie 2.

Dit onderzoek zal zich richten op IT-professionals die betrokken zijn bij Machine Learning Operations, specefiek voor de degenen die het lokaal willen draaien van ML-pipelines.

Het lokaal draaien van machinel learning pipelines met behulp van Kubeflow blijkt een uitdagin te zijn. vooral met betrekking tot de huidige migratieproblemen en gebrekkige documentatie.
De centrale porbleemstelling is daarom: "Hoe kan een ML-pipeline lokaal worden uitgevoerd met behulp van een machine learning framework, rekeninghouden met de uitdagingen van de frameworks?"




%---------- Stand van zaken ---------------------------------------------------

\section{Literatuurstudie}%
\label{sec:state-of-the-art}


%TODO Uitleg over Machine Learning
%TODO uitleg over de Frameworks
%Todo uitleg over CI/CD pipelines
\subsection{Machine Learning}
\subsection{CI/CD pipelines}
\subsection{Frameworks}
\begin{itemize}
    \item Framework 1
    \item Framework 2
    \item Framework 3
\end{itemize}
\subsubsection{Kubeflow}

Hier beschrijf je de \emph{state-of-the-art} rondom je gekozen onderzoeksdomein, d.w.z.\ een inleidende, doorlopende tekst over het onderzoeksdomein van je bachelorproef. Je steunt daarbij heel sterk op de professionele \emph{vakliteratuur}, en niet zozeer op populariserende teksten voor een breed publiek. Wat is de huidige stand van zaken in dit domein, en wat zijn nog eventuele open vragen (die misschien de aanleiding waren tot je onderzoeksvraag!)?

Je mag de titel van deze sectie ook aanpassen (literatuurstudie, stand van zaken, enz.). Zijn er al gelijkaardige onderzoeken gevoerd? Wat concluderen ze? Wat is het verschil met jouw onderzoek?

Verwijs bij elke introductie van een term of bewering over het domein naar de vakliteratuur, bijvoorbeeld~\autocite{Hykes2013}! Denk zeker goed na welke werken je refereert en waarom.

Draag zorg voor correcte literatuurverwijzingen! Een bronvermelding hoort thuis \emph{binnen} de zin waar je je op die bron baseert, dus niet er buiten! Maak meteen een verwijzing als je gebruik maakt van een bron. Doe dit dus \emph{niet} aan het einde van een lange paragraaf. Baseer nooit teveel aansluitende tekst op eenzelfde bron.

Als je informatie over bronnen verzamelt in JabRef, zorg er dan voor dat alle nodige info aanwezig is om de bron terug te vinden (zoals uitvoerig besproken in de lessen Research Methods).

% Voor literatuurverwijzingen zijn er twee belangrijke commando's:
% \autocite{KEY} => (Auteur, jaartal) Gebruik dit als de naam van de auteur
%   geen onderdeel is van de zin.
% \textcite{KEY} => Auteur (jaartal)  Gebruik dit als de auteursnaam wel een
%   functie heeft in de zin (bv. ``Uit onderzoek door Doll & Hill (1954) bleek
%   ...'')

Je mag deze sectie nog verder onderverdelen in subsecties als dit de structuur van de tekst kan verduidelijken.

%---------- Methodologie ------------------------------------------------------
\section{Methodologie}%
\label{sec:methodologie}

Hier beschrijf je hoe je van plan bent het onderzoek te voeren. Welke onderzoekstechniek ga je toepassen om elk van je onderzoeksvragen te beantwoorden? Gebruik je hiervoor literatuurstudie, interviews met belanghebbenden (bv.~voor requirements-analyse), experimenten, simulaties, vergelijkende studie, risico-analyse, PoC, \ldots?

Valt je onderwerp onder één van de typische soorten bachelorproeven die besproken zijn in de lessen Research Methods (bv.\ vergelijkende studie of risico-analyse)? Zorg er dan ook voor dat we duidelijk de verschillende stappen terug vinden die we verwachten in dit soort onderzoek!

Vermijd onderzoekstechnieken die geen objectieve, meetbare resultaten kunnen opleveren. Enquêtes, bijvoorbeeld, zijn voor een bachelorproef informatica meestal \textbf{niet geschikt}. De antwoorden zijn eerder meningen dan feiten en in de praktijk blijkt het ook bijzonder moeilijk om voldoende respondenten te vinden. Studenten die een enquête willen voeren, hebben meestal ook geen goede definitie van de populatie, waardoor ook niet kan aangetoond worden dat eventuele resultaten representatief zijn.

Uit dit onderdeel moet duidelijk naar voor komen dat je bachelorproef ook technisch voldoen\-de diepgang zal bevatten. Het zou niet kloppen als een bachelorproef informatica ook door bv.\ een student marketing zou kunnen uitgevoerd worden.

Je beschrijft ook al welke tools (hardware, software, diensten, \ldots) je denkt hiervoor te gebruiken of te ontwikkelen.

Probeer ook een tijdschatting te maken. Hoe lang zal je met elke fase van je onderzoek bezig zijn en wat zijn de concrete \emph{deliverables} in elke fase?

%---------- Verwachte resultaten ----------------------------------------------
\section{Verwacht resultaat, conclusie}%
\label{sec:verwachte_resultaten}

Hier beschrijf je welke resultaten je verwacht. Als je metingen en simulaties uitvoert, kan je hier al mock-ups maken van de grafieken samen met de verwachte conclusies. Benoem zeker al je assen en de onderdelen van de grafiek die je gaat gebruiken. Dit zorgt ervoor dat je concreet weet welk soort data je moet verzamelen en hoe je die moet meten.

Wat heeft de doelgroep van je onderzoek aan het resultaat? Op welke manier zorgt jouw bachelorproef voor een meerwaarde?

Hier beschrijf je wat je verwacht uit je onderzoek, met de motivatie waarom. Het is \textbf{niet} erg indien uit je onderzoek andere resultaten en conclusies vloeien dan dat je hier beschrijft: het is dan juist interessant om te onderzoeken waarom jouw hypothesen niet overeenkomen met de resultaten.

